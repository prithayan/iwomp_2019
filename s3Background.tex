\subsection{Memory SSA form} Our analysis is based on the LLVM 
 Memory SSA \cite{llvm-memoryssa-url} \cite{Novillo_memoryssa}, 
 which is an imprecise 
implementation of Array SSA\cite{Knobe:1998:ASF:268946.268956}.
The Memory SSA is a virtual IR, that captures the def-use 
information for array variables. Every definition is 
identified by a unique name/number, which is then referenced 
by the corresponding use.
% Memory SSA captures the 
% use-def chains for every memory access in the program. 
% We construct the def-use chains for each array variable, 
% based on the Memory SSA. 
% The store instruction 
% corresponds to the definition, and load instruction corresponds 
% to memory use, and Memory Phi nodes merge the \textit{may} reach 
% definitions. 
% LLVM Memory SSA is a virtual IR, where every definition and phi 
% node creates a new version of memory, which are numbered.

The Memory SSA IR has the following kinds of instructions/nodes, 
\begin{itemize}
% \vspace{-10pt}
 \item $INIT$, a special node to signify uninitialized or live on 
 entry definitions
 \item $N' = MemoryDef(N)$, $N'$ is an operation which may modify memory, 
 and $N$ identifies the last write that $N'$  clobbers.
 \item $MemoryUse(N)$, is an operation that uses the memory written 
 by the definition $N$, and does not modify the memory. 
 \item $MemPhi(N_1,N_2,...)$, is an operation corresponding to a
 basic block, and $N_i$ is one of the may reaching definitions, 
 that could flow into the basic block.
\end{itemize}

We make the following simplifying assumptions, to keep the analysis
tractable
\begin{itemize}
% \vspace{-8pt}
 \item Given an array variable we can find all the  corresponding
 load and store instructions. So, we cannot handle cases, when pointer
 analysis fails to disambiguate the memory a pointer refers to.
 \item A $MemoryDef$ node clobbers the array associated 
 with its store instruction. As a result, write to any array location, 
 is considered to update the entire array.
 \item $MemoryPhi$ nodes are inserted only at the entry 
 of basic blocks, which have more than one $MemoryDef$s that can 
 flow into the basic block.
 \item We analyze only the array variables that 
 are mapped to a target region. 
\end{itemize}

% \vspace{-16pt}
\subsection{Scalar Evolution Analysis}\vspace{-5pt}
LLVM's Scalar Evolution (SCEV) is a very powerful technique
that can be used to analyze the change in 
the value of scalar variables over iterations of a loop, as a chain 
of recurences. 
We can use the SCEV analysis to represent the loop induction variables 
as chain of recurrences. 
This mathematical representation 
can then be used to analyze the variables used to index into memory 
operations. 
Then it can be used to symbolically evaluate the 
minimum and maximum value of every array index expression. 
In case the SCEV analysis fails to model an expression, 
we over approximate the range of the array indices to the 
maximum dimensions of the array variable. 
\autoref{s4} has details 
of how we implement the analysis and handle different cases. 
