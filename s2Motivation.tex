To motivate the utility and applicability of OmpSan,
we discuss 3 potential errors in user code arising from 
improper usage of the data mapping constructs.
% We discuss potential errors in the user code arising 
% from improper usage of the data mapping constructs, 
% and illustrate how easy it is to incorrectly use the map 
% construct. 
% The accompanying examples motivate the utility and applicability of our proposed analysis and the tool OMPSan.

% We show several common pitfalls of the OpenMP data map construct, 
% and illustrate how easy it is to incorrectly use the map 
% construct, and thus motivate the need for our tool.
\vspace{-10pt}
\subsection{Default Scalar Mapping}
In Our first example, \autoref{incorrectegs1},  
 the definition of ``sum'' on line 5 does not reach line 6,
since the ``sum'' does not have an explicit mapping and the default
map for scalars is ``firstprivate''. 
As \autoref{incorrectegs-fix1} shows, an explicit map clause 
is essential to specify the 
copy in and copy out of the scalar ``sum'' from device.

\begin{minipage}{.4\textwidth}
\begin{lstlisting}[style=customc, frame=tlrb, caption={Default scalar map}, label=incorrectegs1]
int A[N], sum=0, i;
#pragma omp target
#pragma omp teams distribute parallel for reduction(+:sum)
  for(i=0; i<N; i++) 
    sum += A[i];
  printf("\n%d",sum);
\end{lstlisting}
\end{minipage}\hfil
\begin{minipage}{.4\textwidth}
\begin{lstlisting}[style=customc, frame=tlrb, caption={Explicit map}, label=incorrectegs-fix1]
int A[N], sum=0;
#pragma omp target map(tofrom:sum)
#pragma omp teams distribute parallel for reduction(+:sum)
  for( int i=0; i<N; i++) 
    sum += A[i];
  printf("\n%d",sum);
\end{lstlisting}
\end{minipage}

\subsection{Reference Count Issues}
\subsubsection{Example 1}
\autoref{incorrectegs3} shows an example of data-mapping 
attributes across different data environments.
The array ``B'', is specified as ``alloc'' in the first 
data environment. According to OpenMP 4.5 (\autoref{mapSemantics}) 
exiting a data environment where the variable was mapped as ``alloc'' does not decrement the reference count, and a variable is mapped 
back from device to host only if the reference count is decremented to 0. 
We can track the reference count for ``B'' is as follows, 
\begin{itemize}
% \vspace*{-10pt}
 \item Line 5, reference count = 1
 \item Line 6, enter data environment, reference count = 2
 \item Line 8, exit data environment ``alloc'', reference count = 2
 \item Line 9, exit data environment ``from'', reference count =1
 \item Line 12 accesses stale data of ``B'', 
 since it was not mapped back to host
\end{itemize}
% 
% variable is set to 1 at line 5, as we enter a new 
% data environment. Then we enter another 
% data environment at line 6, which increments the reference count 
% to 2. But as we exit the environment at line 8, 
% because of the ``alloc'' attribute, the reference count of variable ``B'' 
% is not decremented, so at line 8, reference count is 8, and line 9
% ends the data environment, and has a ``from'' attribute
% which decrements the reference count to 1. Since the reference count 
% is not zero, array ``B'' is not mapped back to host.
% By specifying the ``from'' attribute on the map clause for the 
% variable ``B'' on line 9, 
% the user would hope to see the updated value of B from the device 
% to be available on the host as well. 
% However, such an expectation would be incorrect, and 
% easily missed by the user owing to reference-count rules
% that guide data copy-in/out outcomes. As such corresponding 
% value of ``B'' on host is not updated.
As \autoref{incorrectegs-fix3} shows, replacing ``alloc'' 
with ``from'' on line 6, will update the host version of ``B'' 
on exit of the map region at line 9.

(Note: This is no longer a bug in OpenMP 5.0, since even ``alloc'' 
 decrements the reference counter)

\begin{minipage}{.4\textwidth}
\begin{lstlisting}[style=customc, frame=tlrb, caption={Usage of alloc}, label=incorrectegs3]
int A[10], B[10];
for (int i =0 ; i < 10 ; i++)
    A[i] = i;

#pragma omp target enter data map(to:A[0:10]) map(alloc:B[0:10])
#pragma omp target map(alloc:B[0:10])
for (int i = 0 ; i < 10; i++)
    B[i] = A[i];
#pragma omp target exit data map(from:B[0:10])

for (int i = 0 ; i < 10; i++)
    printf("%d",B[i]);
\end{lstlisting}
\end{minipage}\hfil
\begin{minipage}{.4\textwidth}
\begin{lstlisting}[style=customc, frame=tlrb, caption={Usage of from}, label=incorrectegs-fix3]
int A[10], B[10];
for (int i =0 ; i < 10 ; i++)
    A[i] = i;

#pragma omp target enter data map(to:A[0:10]) map(alloc:B[0:10])
#pragma omp target map(from:B[0:10]) 
for (int i = 0 ; i < 10; i++)
    B[i] = A[i];
#pragma omp target exit data map(from:B[0:10])

for (int i = 0 ; i < 10; i++)
    printf("%d",B[i]);
\end{lstlisting}
\end{minipage}

This example shows the difficulty in interpreting an 
independent map construct. 
Especially when we are dealing with the global variables 
and map clauses across different functions, 
maybe even in different files, 
it becomes nearly impossible to understand 
and identify potential incorrect usages of 
the map construct. 
Our static analysis tool can report diagnostics and errors with possible fixes to help the developers in using the data mapping clauses.
% 
% 
% can not only 
% error out on such issues, 
% but also the show debug information 
% to help understand how each map construct 
% is interpreted based on its context.
\subsubsection{Example 2} 
\autoref{incorrectegs2} shows another example of a reference count issue. 
% reference count, user might not get the expected behavior. 
The line, 9 which executes on the host, does not read
the value of ``A'' that was updated on device at line 7. 
This is again because of the ``from'' clause on line 5, increments 
the reference count to 2 on entry, and back to 1 on exit, hence 
after line 7, ``A'' is not copied out to host.
\autoref{incorrectegs-fix2} shows the usage of ``update'' 
to force the copy-out, and read the expected updated ``A'' on line 11.
% incorrect usage of map clause. 
% The user declared the target data environment on line 3, with ``A'' mapped as ``from''. According to the OpenMP 4.5 semantics, the map clause on line 3, 
% will instantiate an uninitialized version of array ``A'' on device, 
% and also associate a reference count with it. The reference count will 
% be set to 1, after the  line 3. Now the map clause on the ``target'' 
% construct, at line 5, will have no affect on the device copy of ``A'', but 
% still it will increment the reference count to 2 at line 5. 
% At the exit of the offloaded loop, after line 7, the reference count 
% is 2, hence the ``from'' clause on line 5 will not have any affect, 
% other than decrementing the reference count to 1. 
% Now the line 9, that is executed on the host, will not 
% be reading the updated version of ``A'' from the device, line 7, 
% because ``A'' was not mapped back to the host after line 7.
% On exit of the data map region at line 10, the reference count is 
% decremented to 0, and only then the device copy of ``A'' is 
% mapped back and copied to the host. This is because of the 
% ``from'' map attribute on line 3. In this example, ``from'' 
% attribute on line 5, has no affect. 
% To fix this issue, we need to use the update clause as shown in 
% \autoref{incorrectegs-fix2}.

\begin{minipage}{.4\textwidth}
\begin{lstlisting}[style=customc, frame=tlrb,  caption={Reference Count}, label=incorrectegs2]
define N 100                                                                                            
int A[N], sum=0;
#pragma omp target data map(from:A[0:N]) 
  {
    #pragma omp target map(from:A[0:N])
    for(int i=0; i<N; i++) 
      A[i]=i;
    for(int i=0; i<N; i++) 
      sum += A[i];
  }
\end{lstlisting}
\end{minipage}\hfil
\begin{minipage}{.4\textwidth}
\begin{lstlisting}[style=customc, frame=tlrb, caption={Update Clause}, label=incorrectegs-fix2]

define N 100                                                                                            
int A[N], sum=0;
#pragma omp target data map(from:A[0:N]) 
  {
    #pragma omp target map(from:A[0:N])
    for(int i=0; i<N; i++) 
      A[i]=i;
    #pragma omp target update from(A[0:N])  
    for(int i=0; i<N; i++) 
      sum += A[i];
  }
\end{lstlisting}
\end{minipage}
% In the next section, we will introduce our static analysis tool, 
% which can identify such issues with the incorrect/unexpected 
% usage of OpenMP data mapping constructs. 
% Assuming the correct/expected
% behavior is same as when executing the OpenMP program by ignoring all the 
% OpenMP constructs.
% Lets look at an example to motivate that determining when to insert the
% memory copies is non-trivial and can result in incorrect programs, 
% if the programmer does not understand the OpenMP spec precisely. 
% \autoref{datamapping1} shows an example of incorrect usage of the openmp data mapping clause. 
% On line 4, the map clause, specifies \textit{tofrom}, hence at the entry and 
% exit of the omp region, host-device and device-host memory copy is introduced. 
% That is, a new device copy for the variable $sum$ is made at line   4, and 
% the host $sum$ is copied to the device $sum$. Line 6, increments the value to 1. 
% Line 7, where the region ends, copies back the device $sum$ to host $sum$, and hence
% line 8 prints the correct value of $sum$. Again Line 9, has the map clause \textit{to}, 
% which copies the host $sum$ to device $sum$ on line 10, but since \textit{from} is missing
% fromt he clause, the host copy is not updated on line 12, since the device to host memory copy
% is not present. Since, the host does not access, $sum$ on line 13, this is a valid behaviour. 
% But again on line 14, \textit{tofrom} maptype is used, but this time, since the $sum$ 
% was already present on the device, a host device memory copy is not inserted. Also, 
% since the reference count for $sum$ on device is now 2, and the \textit{from} maptype 
% only decrements the reference count to 1, the device-host copy is also not inserted on line 17. 
% This is a common mistake user can make, if he does not carefully follow the OpenMP 4.5 specs.
% 
% \begin{lstlisting}[style=customc, caption={data map clause}, label=datamapping1]
% int main()
% {
%     int sum = 0 ;
%     #pragma omp target data map (tofrom:sum)
%     {  /*H->D*/
%         sum++;  /* sum = 1 */
%     } /*D->H*/
%     printf("%d",sum); /* 1 */
%     #pragma omp target data map (to:sum)  
%     {  /*H->D*/
%         sum++;  /* sum = 2 */
%     }    
%     /* printf("%d",sum);  1 */
%     #pragma omp target data map (tofrom:sum) 
%     {
%         sum++;  /* sum = 3 */  
%     } 
%     printf("%d",sum); /* 3 */
%     return 0;
% }
% \end{lstlisting}


% \begin{lstlisting}[style=customc, caption={data map clause}, label=datamapping1]
%  static void init_ui(int d1, int d2, int d3)
% {
%   int i, j, k;
% 
% #pragma omp target map (alloc: u0_real,u0_imag,u1_real,u1_imag,twiddle)                                                                                                                                            
% //#pragma omp target update to(u0_real,u0_imag,u1_real,u1_imag,twiddle)
%   {
% //#pragma omp target 
% #pragma omp teams distribute 
%     for (k = 0; k < d3; k++) {
% #pragma omp parallel for 
%       for (j = 0; j < d2; j++) {
%         for (i = 0; i < d1; i++) {
%           u0_real[k*d2*(d1+1) + j*(d1+1) + i] = 0.0; 
%           u0_imag[k*d2*(d1+1) + j*(d1+1) + i] = 0.0; 
%           u1_real[k*d2*(d1+1) + j*(d1+1) + i] = 0.0; 
%           u1_imag[k*d2*(d1+1) + j*(d1+1) + i] = 0.0; 
%           twiddle[k*d2*(d1+1) + j*(d1+1) + i] = 0.0; 
%         }
%       }    
%     }    
%   }
% }
% static void evolve(int d1, int d2, int d3)
% {
%   int i, j, k;
% #pragma omp target map (alloc: u0_real,u0_imag,u1_real,u1_imag,twiddle)
%   {
% #pragma omp teams distribute
%     for (k = 0; k < d3; k++) {
% #pragma omp parallel for
%       for (j = 0; j < d2; j++) {
% #pragma omp simd
%         for (i = 0; i < d1; i++) {
%           u0_real[k*d2*(d1+1) + j*(d1+1) + i] = u0_real[k*d2*(d1+1) + j*(d1+1) + i]*twiddle[k*d2*(d1+1) + j*(d1+1) + i];
%           u0_imag[k*d2*(d1+1) + j*(d1+1) + i] = u0_imag[k*d2*(d1+1) + j*(d1+1) + i] *twiddle[k*d2*(d1+1) + j*(d1+1) + i];
%           u1_real[k*d2*(d1+1) + j*(d1+1) + i] = u0_real[k*d2*(d1+1) + j*(d1+1) + i];
%           u1_imag[k*d2*(d1+1) + j*(d1+1) + i] = u0_imag[k*d2*(d1+1) + j*(d1+1) + i];
%         }
%       }
%     }
%   }
% }
% ft.c:#pragma omp target map (alloc: u0_real,u0_imag,u1_real,u1_imag,twiddle)
% ft.c:#pragma omp target map (alloc: u0_real,u0_imag,u1_real,u1_imag,twiddle)
% ft.c:#pragma omp target map(alloc:twiddle)
% ft.c:#pragma omp target map(alloc: u_real,u_imag) 
% ft.c:#pragma omp target map(alloc: u_real,u_imag) map(to:t)
% ft.c:#pragma omp target map(alloc: gty1_real,gty1_imag,gty2_real,gty2_imag, u1_real, u1_imag)\
% ft.c:#pragma omp target map ( alloc: u1_real, u1_imag, u_real,u_imag)\
% ft.c://#pragma omp target map(alloc: gty1_real,gty1_imag,gty2_real,gty2_imag,\
% ft.c://#pragma omp target map(alloc: gty1_real,gty1_imag,gty2_real,gty2_imag,\
% ft.c:#pragma omp target map (alloc: u1_real, u1_imag, u_real,u_imag, gty1_real, gty1_imag, gty2_real, gty2_imag) 
% ft.c://#pragma omp target map(alloc: gty1_real,gty1_imag,gty2_real,gty2_imag,\
% ft.c:#pragma omp target map (alloc: u1_real, u1_imag, u_real,u_imag, gty1_real, gty1_imag, gty2_real, gty2_imag) 
% ft.c:#pragma omp target map (alloc: u1_real, u1_imag) map(tofrom: temp1, temp2)
% \end{lstlisting}
