Managing data transfers to and from GPUs has always been 
an important problem for GPU programming. 
Several solutions have been proposed to help 
the programmer in managing the data movement.
CGCM \cite{Jablin:2011:ACC:1993316.1993516} was one of the first
systems with static analysis to manage CPU-GPU communications. 
It was followed by \cite{Jablin:2012:DMD:2259016.2259038}, a 
dynamic tool for automatic   
CPU-GPU data management. 
The OpenMPC compiler \cite{Lee:2010:OEO:1884643.1884674} also 
proposed a static analysis to insert data transfers automatically.
Seyong et. al proposed a directive based approach, that 
combined compile-time/runtime 
method to verify the correctness of CPU-GPU memory transfer
and even optimize it in \cite{6877281}.
Pai et. al \cite{Pai:2012:FEA:2370816.2370824} proposed a compiler analysis to 
detect potential stale accesses and uses a runtime to initiate transfers as necessary, for the X10 compiler. \cite{Thangamani:2018:ORD:3243176.3243209} 
also proposed a static analysis technique to optimize 
the data transfers for the X10.
\cite{Mendonca:2017:DAA:3086564.3084540} has also worked on automatically 
inferring the OpenMP mapping clauses using some static analysis. 


In this paper, we have developed OMPSan, a static analysis 
tool to interpret the semantics of the openmp map clause, 
and deduce the data transfers introduced by the clause.
It validates, if the data mapping respects the 
original def-use chains of the baseline program. 
Finally OMPSan reports diagnostics, to help 
the developer debug and understand the usage of \texttt{map}
clauses of their program. 

% Then our analysis validates how the def-use chains 
% of the baseline program are modified by the data mapping, 
% and reports if it is not as expected.
