\vspace*{-5pt}
Managing data transfers to and from GPUs has always been 
an important problem for GPU programming. 
Several solutions have been proposed to help 
the programmer in managing the data movement.
CGCM \cite{Jablin:2011:ACC:1993316.1993516} was one of the first
systems with static analysis to manage CPU-GPU communications. 
It was followed by \cite{Jablin:2012:DMD:2259016.2259038}, a 
dynamic tool for automatic   
CPU-GPU data management. 
The OpenMPC compiler \cite{Lee:2010:OEO:1884643.1884674} also 
proposed a static analysis to insert data transfers automatically.
\cite{6877281} proposed a directive based approach for specifying CPU-GPU memory transfers, which included compile-time/runtime methods to verify the correctness of the directives and also identified opportunities for performance optimization.
\cite{Pai:2012:FEA:2370816.2370824} proposed a compiler analysis to 
detect potential stale accesses and uses a runtime to initiate transfers as necessary, for the X10 compiler. \cite{Thangamani:2018:ORD:3243176.3243209} 
also proposed a static analysis technique to optimize 
the data transfers for the X10.
\cite{Mendonca:2017:DAA:3086564.3084540} has also worked on automatically 
inferring the OpenMP mapping clauses using some static analysis. 


In this paper, we have developed OMPSan, a static analysis 
tool to interpret the semantics of the OpenMP map clause, 
and deduce the data transfers introduced by the clause. 
Our algorithm tracks the reference count for individual variables 
to infer the effect of the data mapping clause on the host and device data 
environment.  
We have developed a data flow analysis, on top of LLVM memory 
SSA to capture the def-use information of Array variables. 
We use LLVM Scalar Evolution, to improve the precision of 
our analysis by estimating the range of locations accessed 
by a memory access. This enables the OMPSan to handle array sections also. 
Then OMPSan computes how the data mapping clauses modify 
the def-use chains of the baseline program, and use 
this information to 
validate if the data mapping in the OpenMP program respects the 
original def-use chains of the baseline sequential program. 
Finally OMPSan reports diagnostics, to help 
the developer debug and understand the usage of \texttt{map}
clauses of their program. 
We believe the analysis presented in this paper is very powerful and 
can be developed further for data mapping optimizations also.
We also plan to combine our static analysis 
with a dynamic debugging tool, that would enhance the performance 
of the dynamic tool and also address the limitations of the static analysis.
% Then our analysis validates how the def-use chains 
% of the baseline program are modified by the data mapping, 
% and reports if it is not as expected.
