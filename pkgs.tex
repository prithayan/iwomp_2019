% \usepackage[lofdepth,lotdepth]{subfig}
\usepackage{graphicx}
\usepackage{subcaption}
\usepackage{caption}
\usepackage{float}
% \usepackage{subfigure}
\usepackage{algorithm}
\usepackage[noend]{algpseudocode}
\usepackage{listings,chngcntr}
\usepackage{xcolor} % for setting colors
\usepackage{hyperref}
\usepackage{multirow}
\usepackage{array}
\usepackage{calc}
\newcommand{\algorithmautorefname}{Algorithm}
\newcommand{\definitionautorefname}{Definition}
\newcommand*\Let[2]{\State #1 $\gets$ #2}
\newfloat{algorithm}{t}{lop}

\renewcommand\UrlFont{\color{blue}\rmfamily}

% \usepackage[labelfont=bf,font=small,skip=5pt]{caption}
% \usepackage{subcaption}


% set the default code style
\lstset{
    basicstyle=\footnotesize,
    frame=tb, % draw a frame at the top and bottom of the code block
    tabsize=2, % tab space width
    showstringspaces=false, % don't mark spaces in strings
    numbers=left, % display line numbers on the left
    commentstyle=\color{green}, % comment color
    keywordstyle=\color{blue}, % keyword color
    stringstyle=\color{red} % string color
}
\lstdefinestyle{customc}{
%   belowcaptionskip=1\baselineskip,
  breaklines=true,
  frame=L,
  xleftmargin=\parindent,
  language=C,
  showstringspaces=false,
  basicstyle=\scriptsize\ttfamily,
  keywordstyle=\bfseries\color{green!40!black},
  commentstyle=\itshape\color{purple!40!black},
  identifierstyle=\color{blue},
  stringstyle=\color{orange},
}
\lstdefinestyle{customcnonum}{
%   belowcaptionskip=1\baselineskip,
  breaklines=true,
  frame=L,
  xleftmargin=\parindent,
  language=C,
  numbers=none,
  showstringspaces=false,
  basicstyle=\tiny\ttfamily,
  keywordstyle=\bfseries\color{green!40!black},
  commentstyle=\itshape\color{purple!40!black},
  identifierstyle=\color{blue},
  stringstyle=\color{orange},
}
