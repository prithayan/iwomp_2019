% 
% \begin{minipage}{.45\textwidth}
% \begin{lstlisting}[style=customcnonum, frame=tlrb, caption={DRACC File 22}, label=eval-f22]
%  15 int init(){
%  16     for(int i=0; i<C; i++){
%  17         for(int j=0; j<C; j++){
%  18             b[j+i*C]=1;
%  19         }
%  20         a[i]=1;
%  21         c[i]=0;
%  22     }
%  23         return 0;
%  24 }
%  25                                                                                                                                                                                                                
%  26 
%  27 int Mult(){
%  28 
%  29     #pragma omp target map(to:a[0:C]) map(tofrom:c[0:C]) map(alloc:b[0:C*C]) device(0)
%  30     {
%  31         #pragma omp teams 
%                     distribute parallel for
%  32         for(int i=0; i<C; i++){
%  33             for(int j=0; j<C; j++){
%  34                 c[i]+=b[j+i*C]*a[j];
% 
% \end{lstlisting}
% \end{minipage} \hfill 
% \begin{minipage}{.45\textwidth}
% \begin{lstlisting}[style=customcnonum, frame=tlrb, caption={DRACC File 26}, label=eval-f26]
%  29     #pragma omp target 
%                 enter data map(to:a[0:C],b[0:C*C],c[0:C]) device(0)
%  30     #pragma omp target device(0)
%  31     {
%  32         #pragma omp teams 
%                         distribute parallel for
%  33         for(int i=0; i<C; i++){
%  34             for(int j=0; j<C; j++){
%  35                 c[i]+=b[j+i*C]*a[j];
%  36             }
%  37         }
%  38     }        
%  39     #pragma omp target exit
%                 data map(release:c[0:C]) 
%                 map(release:a[0:C],b[0:C*C]) device(0)
%  40     return 0;
%  41 }
%  42 
%  43 int check(){
%  44     bool test = false;
%  45     for(int i=0; i<C; i++){
%  46         if(c[i]!=C){
% 
% \end{lstlisting}
% \end{minipage}
%  
%  
%  \begin{minipage}{.45\textwidth}
% \begin{lstlisting}[style=customcnonum, frame=tlrb, caption={DRACC File 23}, label=eval-f23]
% 28 int Mult(){
%  29     
%  30     #pragma omp target map(to:a[0:C],b[0:C]) map(tofrom:c[0:C]) device(0)
%  31     {
%  32         #pragma omp teams 
%                         distribute parallel for
%  33         for(int i=0; i<C; i++){
%  34             for(int j=0; j<C; j++){
%  35                 c[i]+=b[j+i*C]*a[j];
%  36             }
%  37         }
%  38     }  
% \end{lstlisting}
% \end{minipage}\hfill 
% \begin{minipage}{.45\textwidth}
% \begin{lstlisting}[style=customcnonum, frame=tlrb, caption={DRACC File 30}, label=eval-f30]
% 19 int init(){
%  20     for(int i=0; i<C; i++){
%  21         for(int j=0; j<C; j++){
%  22             b[j+i*C]=1;
%  23         }
%  24         a[i]=1;
%  25         c[i]=0;                                                                                      
%  26     }
%  ~ 
%  31 int Mult(){
%  32     
%  33     #pragma omp target 
%             map(to:a[0:C],b[0:C*C]) map(from:c[0:C*C]) device(0)
%  34     {
%  35         #pragma omp teams 
%                         distribute parallel for
%  36         for(int i=0; i<C; i++){
%  37             for(int j=0; j<C; j++){
%  38                 c[i]+=b[j+i*C]*a[j];
% \end{lstlisting}
% \end{minipage} 
\
For evaluating OMPSan we use the DRACC\cite{dracc-benchmark} suite, which is a 
benchmarks for data race detection on accelerators, and
also includes several data mapping errors also.
% from Aachen University 
%to evaluate our tool. 
\autoref{dracc-errors} shows some distinct errors found 
by our tool in the benchmark\cite{dracc-benchmark} and the examples of
\autoref{s2}. We were able to find all the known 
data mapping errors in the DRACC benchmark. 
% The last 3 rows of the table shows the errors pointed out by our 
% tool, for the examples from the motivation section. 
% \vspace{-20pt}
% For example in file number 22,\autoref{eval-f22} the map clause on 
% line 29 is using the \textit{alloc} attribute while line 34 is 
% actually reading the array b, that was defined on line 18, so 
% our tool points out the error message that the ``to'' clause is mssing.
% For \autoref{eval-f26} ``c'' is updated on the device at line 34, 
% but it is not mapped back, and our error points out that the 
% ``from/update'' clause is missing.
% For \autoref{eval-f30}, ``c'' is initialized on line 25, 
% but the host variable is not mapped to the device, 
% when creating the device environment at line 33. 
% For \autoref{eval-f23}, only ``C'' elements of ``b''
% are mapped to the device, while the actual size of ``b'' 
% is ``C*C''. Our tool is able to track the malloc call 
% to flag this usage as a warning.
\vspace{-20pt}
\begin{table}
    \caption{Errors found in the DRACC Benchmark and other examples}
    \label{dracc-errors}
    \begin{center}
        \scriptsize
        \begin{tabular}{ |m{1.5in} | m{3in} |}
            \hline
           File Name & Error/Warning \\ \hline    
           DRACC File 22  & ERROR Definition of :b on Line:18 is not reachable to Line:34, Missing Clause:to:Line:32 \\ \hline    
%            DRACC File 24 & ERROR Definition of :b on Line:18 is not reachable to Line:34 Missing Clause:to:Line:32 \\ \hline 
           DRACC File 26 & ERROR Definition of :c on Line:35 is not reachable to Line:46 Missing Clause:from/update:Line:44 \\ \hline 
           DRACC File 30 & ERROR Definition of :c on Line:25 is not reachable to Line:38 Missing Clause:to:Line:36 \\ \hline 
           DRACC File 23 & WARNING Line:30 maps partial data of :b smaller than its total size   \\ \hline 
           Example in \autoref{incorrectegs1} & 
           ERROR Definition of :sum on Line:5 is not 
           reachable to Line:6 Missing Clause:from/update:Line:6 \\ \hline  
%            Example in \autoref{incorrectegs3} & 
%            ERROR Definition of :B on Line:8 is not
%            reachable to Line:12 Missing Clause:from/update:Line:10 
%            \\ \hline 
           Example in \autoref{incorrectegs2} &
           ERROR Definition of :A on Line:7 is not 
           reachable to Line:9 Missing Clause:from/update:8
           \\ \hline                              
        \end{tabular}        
    \end{center}
\end{table} \vspace{-30pt}
% \begin{table}
%     \caption{Errors found in the DRACC Benchmark}
%     \label{dracc-errors}
%     \begin{center}
%         \scriptsize
%         \begin{tabular}{ |m{1.5in} | m{3in} |}
%             \hline
%            File Name & Error/Warning \\ \hline    
%            DRACC File 22 \autoref{eval-f22} & ERROR Definition of :b on Line:18 is not reachable to Line:34, Missing Clause:to:Line:32 \\ \hline    
% %            DRACC File 24 & ERROR Definition of :b on Line:18 is not reachable to Line:34 Missing Clause:to:Line:32 \\ \hline 
%            DRACC File 26 \autoref{eval-f26}& ERROR Definition of :c on Line:35 is not reachable to Line:46 Missing Clause:from/update:Line:44 \\ \hline 
%            DRACC File 30 \autoref{eval-f30}& ERROR Definition of :c on Line:25 is not reachable to Line:38 Missing Clause:to:Line:36 \\ \hline 
%            DRACC File 23 \autoref{eval-f23}& WARNING Line:30 maps partial data of :b smaller than its total size   \\ \hline 
%            Motivation Ex \autoref{incorrectegs1} & 
%            ERROR Definition of :sum on Line:5 is not 
%            reachable to Line:6 Missing Clause:from/update:Line:6 \\ \hline  
%            Motivation Ex \autoref{incorrectegs3} & 
%            ERROR Definition of :B on Line:8 is not
%            reachable to Line:12 Missing Clause:from/update:Line:10 
%            \\ \hline 
%            Motivation Ex \autoref{incorrectegs2} &
%            ERROR Definition of :A on Line:7 is not 
%            reachable to Line:9 Missing Clause:from/update:8
%            \\ \hline 
%            \hline                          
%         \end{tabular}        
%     \end{center}
% \end{table}
% The errors were explained in \autoref{s2}
\subsection{Analysis Time}
\autoref{runtime} shows the time to run OMPSan, on few SPEC ACCEL and 
NAS parallel benchmarks. 
% We compare it with the time to compile the programs with ``-O3'' flag. 
Due to the context and flow sensitive data flow analysis implemented in 
OMPSan, 
its analysis time can be significant; 
however the analysis time is less than or equal to the \textit{-O3}
compilation time in all cases.
%its runtime is significant, and almost as much as the 
%time to compile the entire program with ``-O3''. 
\begin{table} \vspace{-30pt}
    \caption{Time to Run OMPSan}
    \label{runtime}
    \begin{center}
        \scriptsize
        \begin{tabular}{ | c | c | c |}
            \hline
           Benchmark Name &  -O3 Compilation Time (sec)
           & OMPSan Runtime (sec) \\ \hline    
           SPEC 504.polbm & 17 & 16 \\ \hline    
           SPEC 503.postencil & 3 & 3 \\ \hline    
           SPEC 552.pep  & 7 & 4 \\ \hline    
           SPEC 554.pcg & 15 & 9\\ \hline    
           NAS FT & 32 & 15 \\ \hline    
           NAS MG & 34 & 31              \\ \hline                       
        \end{tabular}        
    \end{center}
\end{table}\vspace{-40pt}
\subsection{Diagnostic Information}
\label{diagnostic}
Another major use case for OMPSan, is to help OpenMP developers 
understand 
the data mapping behavior of their source code. 
For example, \autoref{nas-ft-1} shows a code fragment 
from the benchmark ``FT'' in the ``NAS'' suite.
% We argue that it is very difficult to read 
% this openmp target code, as line 311 explicitly 
% maps every array variable with the ``alloc'' 
% map type. Since, these are global variables, the 
% semantics of this \texttt{target map} depend 
% on the previous occurrences of the mapping clause. 
Our tool can generate the following information
diagnostic information on the current version 
 of the data mapping clause.
\begin{itemize}
    \vspace{-4pt}
 \item \texttt{$\_\_tgt\_target\_teams$}, from::``ft.c:311'' to ``ft.c:331''
 \item Alloc: $u0\_imag[0:8421376],u0\_real[0:8421376]$
 \item Persistent In :: $twiddle[0:8421376]$, 
 $u1\_imag[0:8421376],u1\_real[0:8421376]$
 \item  Persistent Out ::  $twiddle[0:8421376],u0\_imag[0:8421376],u0\_real[0:8421376],u1\_imag[0:8421376],u1\_real[0:8421376]$
 \item Copy In:: \textit{Null}, Copy Out:: \textit{Null} 
\end{itemize}
% Similarly, the data mapping for the function $cffts3\_pos$ \autoref{nas-ft-2}, is as shown below,
% \begin{itemize}
%     \vspace{-4pt}
%  \item \texttt{$\_\_tgt\_target\_teams$}, from::``ft.c:1180'' to ``ft.c:1276''
%  \item Persistent In ::
%  $gty1\_imag[0:16777216],gty1\_real[0:16777216],gty2\_imag[0:16777216],gty2\_real[0:16777216],logd3[0:0],u0\_imag[0:8421376],u0\_real[0:8421376],u1\_imag[0:8421376],u1\_real[0:8421376],u\_imag[0:257],u\_real[0:257]$
%  \item Persistent Out: $u1\_imag[0:8421376],u1\_real[0:8421376],u\_imag[0:257],u\_real[0:257]$
%  \item Copy Out: $gty1\_imag[0:16777216],gty1\_real[0:16777216],gty2\_imag[0:16777216],gty2\_real[0:16777216],u0\_imag[0:8421376],u0\_real[0:8421376]$
% \end{itemize}
\hspace{-10pt}
\begin{minipage}{1.1\textwidth}
\begin{lstlisting}[style=customcnonum,basicstyle=\scriptsize, frame=tlrb, caption={\textit{evolve} from NAS/ft.c}, label=nas-ft-1]
 307 static void evolve(int d1, int d2, int d3)
 308 {
 309  int i, j, k;
 311 #pragma omp target map (alloc: u0_real,u0_imag,u1_real,u1_imag,twiddle)
 312  {
 313 #pragma omp teams distribute
 314   for (k = 0; k < d3; k++) {
 315 #pragma omp parallel for
 316    for (j = 0; j < d2; j++) {
 317 #pragma omp simd
 318     for (i = 0; i < d1; i++) {
 319      u0_real[ ... ] = u0_real[ ... ]*twiddle[ ... ];
 321      u0_imag[ ... ] = u0_imag[ ... ]*twiddle[ ... ];
\end{lstlisting}
\end{minipage}  
% \hfill 
% \begin{minipage}{.45\textwidth}
% \begin{lstlisting}[style=customcnonum, frame=tlrb, caption={\textit{cffts3\_pos} from NAS/ft.c}, label=nas-ft-2]
%  1153 static void cffts3_pos(int is, int d1, int d2, int d3)
% 1154 {
% 1155   int logd3;
% 1156   int i, j, k, ii;
% 1157   int l, j1, i1, k1;
% 1158   int n1, li, lj, lk, ku, i11, i12, i21, i22;
% 1159   double uu1_real, x11_real, x21_real;
% 1160   double uu1_imag, x11_imag, x21_imag;
% 1161   double uu2_real, x12_real, x22_real;
% 1162   double uu2_imag, x12_imag, x22_imag;
% 1163   double temp_real, temp2_real;
% 1164   double temp_imag, temp2_imag;
% 1165 
% 1166   logd3 = ilog2(d3);
% ~
% 1180 #pragma omp target teams distribute collapse(2)
% 1181     for (j = 0; j < d2; j++) {
% 1182 //#pragma acc loop vector independent
% 1183 //#pragma omp simd
% 1184       for (i = 0; i < d1; i ++) {
% 1185 #pragma omp parallel for
% 1186         for (k = 0; k < d3; k++) {
% 1187           gty1_real[j][k][i] = u1_real[k*d2*(d1+1) + j*(d1+1) + i];
% 1188           gty1_imag[j][k][i] = u1_imag[k*d2*(d1+1) + j*(d1+1) + i];
% 1189         }
% 1190 
% 
% \end{lstlisting}
% \end{minipage} 
\vspace{-10pt}
\subsection{Limitations}
\label{limitation}
Since OMPSan is a static analysis tool, 
it includes a few limitations. OMPSan, 
% as listed below: 
\vspace{-2pt}
\begin{itemize}
 \item Supports statically and dynamically allocated array variables, 
 but cannot 
 handle dynamic data structures like linked lists
%  (Future extension, use shape analysis)
 It can possibly be 
 addressed in future through advanced static analysis techniques (like shape analysis).
 \item Cannot handle target regions inside recursive functions. 
%  It can 
%  be addressed by improving our context 
%  sensitive analysis.
 It can possibly be addressed in future work by improving our context 
 sensitive analysis.
 \item Can only handle compile time 
 constant array sections, and constant loop bounds.
%   it requires support to 
%  compare the equivalence of two symbolic expressions.
%  OMPSan can be extended in the future
 We can handle runtime expressions, by adding static analysis support to 
 compare the equivalence of two symbolic expressions.
 \item Cannot handle “declare target” 
 since it requires analysis across LLVM modules.
 \item May report false positives for irregular array accesses.
  For example, even if only a small section of the array is updated, 
  our analysis may assume the entire array was updated. More 
  expensive analysis like symbolic analysis can be used to 
  improve the precision of the static analysis.
  \item May fail if Clang/LLVM introduces bugs while 
  lowering OpenMP pragmas to the RTL calls in the LLVM IR.
  Since, 
  we rely heavily on static analysis like Memory SSA and SCEV analysis, 
  it is difficult to implement our algorithm on the clang AST.
  \item May report false positives, if the OpenMP program relies 
  on some dynamic reference count mechanism. 
  That is our algorithm may not be able to 
  statically track a variable's reference count, at the data map clause, 
  and the diagnostics will be inaccurate. Runtime debugging approach 
  will be required to handle such cases.
  
%  since it is 
%  moved to a different module in LLVM IR. 
%  since in LLVM IR, 
%  the ``declare'' region is moved to a separate region, this 
%  can also be fixed if we can analyze the modules together. 
\end{itemize}
% 
% Our analysis shows the following error for \autoref{eval-ex1}, 
% that was used as motivation in \autoref{s2},
% \begin{verbatim}
% ERROR Definition of :sum on Line:5 is not 
% reachable to Line:6 Missing Clause:from/update:Line:6 
% \end{verbatim}
% Since the user missed the explicit map clause for ``sum'', 
% the default mapping was first private, and the value was not mapped back to the host, hence line 6 used a stale version of the variable sum.
% % \autoref{data-mapping-example} shows the memory copies deduced by our tool.
% 
% As explained in motivation \autoref{s2}, for \autoref{eval-ex2} our tool is able to catch the data mapping issue, but please note that
% the Openmp 5.0 spec has been modified so that this is no longer 
% a bug according to OpenMP 5.0.
% \begin{verbatim}
%  ERROR Definition of :B on Line:8 is not
%  reachable to Line:12 Missing Clause:from/update:Line:10
% \end{verbatim}
% 
% Our tool is also able to catch the error for \autoref{incorrectegs2}, 
% from \autoref{s2}.
% \begin{verbatim}
% ERROR Definition of :A on Line:7 is not 
% reachable to Line:9 Missing Clause:from/update:8
% \end{verbatim}
% The reason for this error was that, the reference 
% count for variable A is incremented at line 5, 
% after it was set to 1 at line 3,
% and device-host memory copy is inserted only if the reference count 
% is decremented to zero.
% % \autoref{data-mapping-example} shows the reason for this error, A 
% % is not copied out at line 7, hence the read on line 9 was stale. 
% 
% \begin{minipage}{.4\textwidth}
% \begin{lstlisting}[style=customc, frame=tlrb, caption={Wrong Usage of default map on Reduction Scalar}, label=eval-ex1]
% int A[N], sum=0, i;
% #pragma omp target
% #pragma omp teams distribute parallel for reduction(+:sum)
%   for(i=0; i<N; i++) 
%     sum += A[i];
%   printf("\n%d",sum);
% \end{lstlisting}
% \end{minipage}\hfill 
% \begin{minipage}{.4\textwidth}
% \begin{lstlisting}[style=customc, frame=tlrb, caption={Usage of alloc vs from}, label=eval-ex2]
% int A[10], B[10];
% for (int i =0 ; i < 10 ; i++)
%     A[i] = i;
% 
% #pragma omp target enter data map(to:A[0:10]) map(alloc:B[0:10])
% #pragma omp target map(alloc:B[0:10])
% for (int i = 0 ; i < 10; i++)
%     B[i] = A[i];
% #pragma omp target exit data map(from:B[0:10])
% 
% for (int i = 0 ; i < 10; i++)
%     printf("%d",B[i]);
% \end{lstlisting}
% \end{minipage} 
% 
% 
%  


% \begin{table}
%     \caption{Compile Time for our Tool}
%     \label{dracc-errors}
%     \begin{center}
%         \scriptsize
%         \begin{tabular}{ |m{0.7in} | m{0.6in} | m{0.7in} |}
%             \hline
%            Benchmark Name & Analysis Runtime (sec) & 
%            Standard Compile Time (sec) \\ \hline    
%            spec accel, 504 polbm & ~16.3 & ~0.15 \\ \hline    
% 
%             \hline               
%            \hline                          
%         \end{tabular}        
%     \end{center}
% \end{table}

% 
% \begin{minipage}{.45\textwidth}
% \begin{lstlisting}[style=customc, frame=tlrb, caption={explicit map}, label=eval-ex2]
% int A[N], sum=0, i;
% #pragma omp target map(tofrom:sum)
% #pragma omp teams distribute parallel for reduction(+:sum)
%   for(i=0; i<N; i++) 
%     sum += A[i];
%   printf("\n%d",sum);
% \end{lstlisting}
% \end{minipage}
% \begin{minipage}{.4\textwidth}
% \begin{lstlisting}[style=customc, frame=tlrb,  caption={Reference Count}, label=eval-ex3]
% define N 100                                                                                            
% int A[N], sum=0;
% #pragma omp target data map(from:A[0:N]) 
%   {
%     #pragma omp target map(from:A[0:N])
%     for(int i=0; i<N; i++) 
%       A[i]=i;
%     for(int i=0; i<N; i++) 
%       sum += A[i];
%   }
% \end{lstlisting}
% \end{minipage}\hfil
% \begin{minipage}{.4\textwidth}
% \begin{lstlisting}[style=customc, frame=tlrb, caption={Update Clause}, label=eval-ex4]
% define N 100                                                                                            
% int A[N], sum=0;
% #pragma omp target data map(from:A[0:N]) 
%   {
%     #pragma omp target map(from:A[0:N])
%     for(int i=0; i<N; i++) 
%       A[i]=i;
%     #pragma omp target update from(A[0:N])  
%     for(int i=0; i<N; i++) 
%       sum += A[i];
%   }
% \end{lstlisting}
% \end{minipage}
% \begin{table}
%     \caption{Output Data mapping }
%     \scriptsize
%     \label{data-mapping-example}
%     \begin{center}
% %         \footnotesize
%         \begin{tabular}{|c |c | m{0.5in} | m{0.4in} | m{0.5in} | m{0.6in} 
%                 | m{0.5in} | m{0.6in} | m{0.5in} |}
%             \hline
%            Listing Name & RTL name & Region Line Begin & 
%            Region Line end & Copy In & Persistent In 
%            & Copy Out & Persistent out & Alloc \\ \hline    
%            \autoref{eval-ex1} & $\_\_tgt\_target\_teams$  
%            & 2 & 5 & A[0:100], sum & & A[0:100]& sum & \\ \hline
%            \autoref{eval-ex2} & $\_\_tgt\_target\_teams$  
%            & 2 & 5 & A[0:100], sum & & A[0:100], sum& & \\ \hline
%            \autoref{eval-ex3} & $\_\_tgt\_data\_begin$  
%            & 3 & 10 &  & & A[0:100] & &A[0:100]  \\ \hline
%            \autoref{eval-ex3} & $\_\_tgt\_target$  
%            & 5 & 7 &  & A[0:100] & &A[0:100] & \\ \hline
%            \autoref{eval-ex4} & $\_\_tgt\_data\_begin$  
%            & 3 & 10 &  & & A[0:100] & &A[0:100]  \\ \hline
%            \autoref{eval-ex4} & $\_\_tgt\_target$  
%            & 5 & 7 &  & A[0:100] & &A[0:100] & \\ \hline
%            \autoref{eval-ex4} & $\_\_tgt\_target\_data\_update$  
%            & 9 & 9 &  & & A[0:100] & & \\ \hline
%            
% %            $\_\_tgt\_target$  
% %            & 7 & 10 &  & A[0:10], B[0:10] & A[0:10] & B[0:10] & \\ \hline               
% %            $\_\_tgt\_target\_data\_end$  
% %            & 10 & 10 &  & & B[0:10] & & \\ \hline               
%            \hline                          
%         \end{tabular}        
%     \end{center}
% \end{table}

